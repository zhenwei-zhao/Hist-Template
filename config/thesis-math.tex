%====================== 数学环境 / 定理 / 算法 ======================%
\usepackage{amsmath,amssymb,amsthm}
\usepackage{mathtools}
\usepackage{bm}
\usepackage{cases}
\usepackage{cancel}

% 定理环境
% plain:定理类(定理/引理/命题/推论)内容为斜体
% 注意:中文字体通常没有真正的 Italic,这里的“斜体”表现为倾斜,属于正常现象。
\theoremstyle{plain}
\newtheorem{theorem}{定理}[chapter]
\newtheorem{lemma}{引理}[chapter]
\newtheorem{proposition}{命题}[chapter]
\newtheorem{corollary}{推论}[chapter]

% definition:定义类(定义/例)内容为正体
\theoremstyle{definition}
\newtheorem{definition}{定义}[chapter]
\newtheorem{example}{例}[chapter]

% remark:说明类(注)内容为正体
\theoremstyle{remark}
\newtheorem{remark}{注}[chapter]

\makeatletter
\renewenvironment{proof}[1][\proofname]{%
  \par\pushQED{\qed}%
  \normalfont
  \topsep6\p@\@plus6\p@\relax
  \trivlist
  \item[\hskip\labelsep
        \bfseries #1\@addpunct{.}]%
}{%
  \popQED\endtrivlist\@endpefalse
}
\makeatother



% 编号:按章
\numberwithin{equation}{chapter}
\numberwithin{figure}{chapter}
\numberwithin{table}{chapter}

% 图表/公式编号格式:章-序(如 1-1)
\renewcommand{\thefigure}{\arabic{chapter}-\arabic{figure}}
\renewcommand{\thetable}{\arabic{chapter}-\arabic{table}}
\renewcommand{\theequation}{\arabic{chapter}-\arabic{equation}}
\renewcommand{\figurename}{图}
\renewcommand{\tablename}{表}

%====================== 算法(algorithm) ======================%
\usepackage{algorithm}
\usepackage{algpseudocode}
\floatname{algorithm}{算法}
\algrenewcommand\algorithmicrequire{\textbf{输入:}}
\algrenewcommand\algorithmicensure{\textbf{输出:}}
