
%====================== 字体设置(中英文) ======================%
% 中文字体:
%   - main.tex 使用 fontset=fandol(TeX Live 自带),跨平台最稳定。
%   - 因此这里不再手动 \setCJKmainfont,避免与 ctex fontset 冲突。
%
% 英文字体(建议使用 TeX Live 自带字体,避免系统缺字导致编译失败):
%   - Times 风格:TeX Gyre Termes
%   - Helvetica 风格:TeX Gyre Heros
%   - Courier 风格:TeX Gyre Cursor
%
% 若你在 Windows 且希望使用 Times New Roman/Arial/Courier New,可将下面三行替换为:
  % \setmainfont{Times New Roman}
  % \setsansfont{Arial}
  % \setmonofont{Courier New}

% \setmainfont{TeX Gyre Termes}
% \setsansfont{TeX Gyre Heros}
% \setmonofont{TeX Gyre Cursor}

%====================== 英文字体:自动适配 ======================%
% \usepackage{fontspec}

% \IfFontExistsTF{Times New Roman}{
%   \setmainfont{Times New Roman}
% }{
%   \setmainfont{TeX Gyre Termes}
% }

% \IfFontExistsTF{Arial}{
%   \setsansfont{Arial}
% }{
%   \setsansfont{TeX Gyre Heros}
% }

% \IfFontExistsTF{Courier New}{
%   \setmonofont{Courier New}
% }{
%   \setmonofont{TeX Gyre Cursor}
% }

% 加载文件包中的字体
% =========================
% Local fonts config (./fonts)
% Compile with XeLaTeX or LuaLaTeX
% =========================
\usepackage{fontspec}

% 让 fontspec 可以直接按文件名找字体(不依赖系统安装)
\defaultfontfeatures{Ligatures=TeX}

% ---------- 英文字体(来自 ./fonts) ----------
\setmainfont{TimesNewRoman.ttf}[
  Path = ./fonts/,
  UprightFont = TimesNewRoman.ttf,
  BoldFont    = timesbd.ttf,
  ItalicFont  = timesi.ttf,
  BoldItalicFont = timesbi.ttf
]
% 如果你后续需要英文无衬线,可再补 Arial.ttf;目前你目录里没有,就先不设
% \setsansfont{Arial.ttf}[Path=./fonts/]

\setmonofont{CourierNew.ttf}[
  Path = ./fonts/,
  UprightFont = CourierNew.ttf
]

% ---------- 中文字体(来自 ./fonts) ----------
% 正文中文:宋体;加粗用 SimSunB.TTF
\setCJKmainfont{SimSun.ttf}[
  Path = ./fonts/,
  UprightFont = SimSun.ttf,
  % BoldFont    = SimHei.TTF,
    AutoFakeBold = 2.0,
   AutoFakeSlant = 0.0
]

\setCJKmonofont{SimSun.ttf}[
    Path=./fonts/, 
    UprightFont=SimSun.ttf, 
    AutoFakeBold=2.0
]

% 黑体:用于标题等(如果你模板已定义 \heiti,这里也可以用下面的 \hei)
\setCJKsansfont{SimHei.ttf}[
  Path = ./fonts/,
  UprightFont = SimHei.ttf,
  AutoFakeBold = 2.0
]

% ---------- 额外中文字体族:仿宋、楷体 ----------
\setCJKfamilyfont{fs}{FangSong.ttf}[
  Path = ./fonts/,
  UprightFont = FangSong.ttf
]
\setCJKfamilyfont{kai}{SimKai.ttf}[
  Path = ./fonts/,
  UprightFont = SimKai.ttf
]
\setCJKfamilyfont{hei}{SimHei.ttf}[
  Path = ./fonts/,
  UprightFont = SimHei.ttf,
  AutoFakeBold = 2.0
]
\setCJKfamilyfont{song}{SimSun.ttf}[
  Path = ./fonts/,
  UprightFont = SimSun.ttf,
  AutoFakeBold = 2.0,
  % BoldFont    = SimSunBold.ttf
]

% 方便你在正文里随时切换:
\newcommand{\fs}{\CJKfamily{fs}}     % 仿宋
\newcommand{\kai}{\CJKfamily{kai}}   % 楷体
\newcommand{\hei}{\CJKfamily{hei}}   % 黑体
\newcommand{\song}{\CJKfamily{song}} % 宋体



