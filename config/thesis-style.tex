
%====================== 常用宏包(排版/图表/列表) ======================%
\usepackage{graphicx}
\usepackage{booktabs}
\usepackage{array}
\usepackage{multirow}
\usepackage{longtable}
\usepackage{tabularx}
\usepackage{makecell}
\usepackage{threeparttable}
\usepackage{threeparttablex}
\usepackage{diagbox}
\usepackage{rotating}
% 浮动控制(可选;正文尽量不用 [H])
\usepackage{float}
\usepackage{placeins}
\usepackage{enumitem}
\usepackage{caption}
\usepackage{listings}
\usepackage{xcolor}
\usepackage{subcaption}
\setlist[itemize]{itemsep=0.2em, topsep=0.4em}
\setlist[enumerate]{itemsep=0.2em, topsep=0.4em}



%====================== 固定长度下划线(封面/信息页) ======================%
\newcommand{\fixuline}[2][4cm]{%
  \makebox[#1][c]{\underline{\makebox[#1][c]{#2}}}%
}
%====================== caption(图/表标题) ======================%
\captionsetup[figure]{font=small, labelfont=bf, labelsep=quad}
\captionsetup[table]{font=small, labelfont=bf, labelsep=quad}

%====================== 章节标题格式(ctex) ======================%
\ctexset{
  chapter = {
    format = \centering\hei\zihao{3},
    name = {第,章},
    number = \chinese{chapter},
    aftername = \ ,
    beforeskip = -1em,
    afterskip = 1em
  },
  section = {
    format = \hei\zihao{4},
    aftername = \ ,
    beforeskip = 20pt,
    afterskip = 10pt
  },
  subsection = {
    format = \hei\zihao{4},
    aftername = \ ,
    beforeskip = 0.5em,
    afterskip = 0.5em
  },
  subsubsection = {
    format = \hei\zihao{4},
    aftername = \ ,
    beforeskip = 8pt,
    afterskip  = 4pt
  },
}

% % 三级标题编号深度
\setcounter{secnumdepth}{4}

%====================== 工具:插入空白页(双面排版补空页) ======================%
\newcommand{\blankpage}{%
  \clearpage
  \ifodd\value{page}\else
    \thispagestyle{empty}\null\newpage
  \fi
}

%====================== 页眉页脚(含章节首页) ======================%
% empty:页眉和页脚都为空。
% plain:页眉为空,页脚包含一个居中的页码(这是标准文档类的默认样式)。
% headings:页眉包含章节标题和页码,页脚为空。
% myheadings:类似headings,但内容可由用户自定义。
% fancy:使用fancyhdr宏包定义的更灵活的页眉页脚样式。

% \fancyhead[L]{左页眉内容}  % 页眉左侧
% \fancyhead[C]{中间页眉内容} % 页眉中间
% \fancyhead[R]{右页眉内容}  % 页眉右侧
% \fancyfoot[L]{左页脚内容}  % 页脚左侧
% \fancyfoot[C]{中间页脚内容} % 页脚中间
% \fancyfoot[R]{右页脚内容}  % 页脚右侧

% \fancyhead{}                     % 清空页眉
% \fancyfoot{}                     % 清空页脚
% \fancyhf{}   % 清空页脚和页眉

\usepackage{fancyhdr}
\renewcommand{\MakeUppercase}[1]{#1}
\fancyhf{}  %用于清空(重置)当前所有的页眉页脚设置。
\pagestyle{fancy}
% % 页眉内容:论文中文题目(可在 config/thesis-info.tex 修改)
\fancyhead[C]{\HISTTitleCN}  % 页眉 居中
\fancyfoot[C]{\song\zihao{5}\thepage}  % 页脚为页码  居中
\renewcommand{\headrulewidth}{0.4pt}   % 设置页眉横线宽度
\renewcommand{\footrulewidth}{0pt}     % 设置页脚横线宽度


% 章节首页(plain)有页眉页脚与普通页保持一致   plain 默认样式
\fancypagestyle{plain}{
  \fancyhf{}
  % \fancyhead[C]{\HISTTitleCN}
  % \fancyhead[C]{\zihao{5}\song\rightmark}
  \fancyhead[C]{\zihao{5}\song\leftmark}
  \fancyfoot[C]{\song\zihao{5}\thepage}
  \renewcommand{\headrulewidth}{0.4pt}
  \renewcommand{\footrulewidth}{0pt}
}

% \makeatletter
% % 章标题写入 \leftmark(用于页眉显示章)
% \renewcommand{\chaptermark}[1]{%
%   \markboth{第\thechapter 章\quad #1}{}%
% }
% % 节标题写入 \rightmark(如果你需要显示节)
% \renewcommand{\sectionmark}[1]{%
%   \markright{\thesection\quad #1}%
% }
% \makeatother


% 前置部分专用:无页眉有页脚  自定义样式
\fancypagestyle{plain-front}{
  \fancyhf{}
  \fancyfoot[C]{\song\zihao{5}\thepage}
  \renewcommand{\headrulewidth}{0pt}
  \renewcommand{\footrulewidth}{0pt}
}

\fancypagestyle{plain-none}{
  \fancyhf{}
  \fancyfoot[C]{\song\zihao{5}\thepage}
  \renewcommand{\headrulewidth}{0pt}
  \renewcommand{\footrulewidth}{0pt}
}


%====================== 表格常用列类 ======================%
\newcolumntype{Y}{>{\centering\arraybackslash}X}
\newcolumntype{P}[1]{>{\raggedright\arraybackslash}p{#1}}
\newcolumntype{C}[1]{>{\centering\arraybackslash}p{#1}}
\newcolumntype{R}[1]{>{\raggedleft\arraybackslash}p{#1}}

%====================== 目录格式(小四 + 1.5倍行距)======================%

\usepackage{tocloft}

\renewcommand{\contentsname}{目\quad 录}

\makeatletter
\renewcommand{\@cftmaketoctitle}{%
  \addpenalty\@secpenalty
  \if@cfthaschapter
    \vspace*{\cftbeforetoctitleskip}%
  \else
    \vspace{\cftbeforetoctitleskip}%
  \fi
  \@cftpagestyle
  {\interlinepenalty\@M
   \centering\hei\bfseries\zihao{-3}\contentsname\par
   \cftmarktoc
   \par\nobreak
   \vskip \cftaftertoctitleskip
   \@afterheading}}
\makeatother

\renewcommand{\cftdotsep}{1}
\setlength{\cftbeforetoctitleskip}{0pt}  % 标题前
\setlength{\cftaftertoctitleskip}{18pt}   % 标题后

% 目录正文统一:宋体 + 小四  章节设置
\renewcommand{\cftchapfont}{\song\bfseries\zihao{-4}}
\renewcommand{\cftsecfont}{\song\zihao{-4}}
\renewcommand{\cftsubsecfont}{\song\zihao{-4}}

\renewcommand{\cftchappagefont}{\song\bfseries\zihao{-4}} 
\renewcommand{\cftsecpagefont}{\song\zihao{-4}}
\renewcommand{\cftsubsecpagefont}{\song\zihao{-4}}
% 各级条目前后间距(一般设 0pt 或很小)
\setlength{\cftbeforechapskip}{0pt}
\setlength{\cftbeforesecskip}{0pt}
\setlength{\cftbeforesubsecskip}{0pt}
% 让“章/节/小节”都使用点线 leader
\renewcommand{\cftchapleader}{\cftdotfill{\cftdotsep}}
\renewcommand{\cftsecleader}{\cftdotfill{\cftdotsep}}
\renewcommand{\cftsubsecleader}{\cftdotfill{\cftdotsep}}

\renewcommand{\listfigurename}{插\quad 图}

\makeatletter
\renewcommand{\@cftmakeloftitle}{%
  \addpenalty\@secpenalty
  \if@cfthaschapter
    \vspace*{\cftbeforeloftitleskip}%
  \else
    \vspace{\cftbeforeloftitleskip}%
  \fi
  \@cftpagestyle
  {\interlinepenalty\@M
   \centering\hei\zihao{-3}\listfigurename\par
   \cftmarklof
   \par\nobreak
   \vskip \cftafterloftitleskip
   \@afterheading}}
\makeatother


% 插图目录标题上下空白
\setlength{\cftbeforeloftitleskip}{0pt}
\setlength{\cftafterloftitleskip}{18pt}


\renewcommand{\listtablename}{表\quad 格}
\makeatletter
\renewcommand{\@cftmakelottitle}{%
  \addpenalty\@secpenalty
  \if@cfthaschapter
    \vspace*{\cftbeforelottitleskip}%
  \else
    \vspace{\cftbeforelottitleskip}%
  \fi
  \@cftpagestyle
  {\interlinepenalty\@M
   \centering\hei\zihao{-3}\listtablename\par
   \cftmarklot
   \par\nobreak
   \vskip \cftafterlottitleskip
   \@afterheading}}
\makeatother


% 表格目录标题上下空白
\setlength{\cftbeforelottitleskip}{0pt}
\setlength{\cftafterlottitleskip}{18pt}

%====================== 目录格式(小四 + 1.5倍行距)======================%
\lstset{
  basicstyle=\ttfamily\small,      % 等宽字体
  % numbers=left,                    % 行号
  numberstyle=\tiny,
  stepnumber=1,
  numbersep=8pt,
  frame=single,                    % 边框
  breaklines=true,                 % 自动换行
  tabsize=3,
  captionpos=b,
  showstringspaces=false,
  keywordstyle=\color{blue},
  commentstyle=\color{gray},
  stringstyle=\color{red}
}

\usepackage{tikz}

\definecolor{kwbg}{RGB}{225,235,250}  % 浅亮蓝
\definecolor{kwtx}{RGB}{30,80,150}    % 深蓝字

\newcommand{\kw}[1]{%
  \tikz[baseline=(X.base)]{
    \node[
      fill=kwbg,
      rounded corners=2pt,
      inner xsep=5pt,
      inner ysep=2pt
    ] (X) {\textsf{\small\textcolor{kwtx}{#1}}};
  }%
}