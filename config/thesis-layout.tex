
%====================== 页面与行距 ======================%
\usepackage{geometry}
\geometry{
  left=3.18cm, right=3.18cm, top=2.54cm, bottom=2.54cm,
  headsep=0.8cm,
  headheight=15pt
}

% 建议:允许页面底部不强行拉伸,减少 Underfull \vbox 警告
\raggedbottom

%====================== 正文段落(字号与行距) ======================%
% 说明:
%  - main.tex 采用 zihao=-4(小四,约 12pt)。
%  - 如学校要求“固定行距”,推荐用 \normalsize 统一设定,避免 \baselineskip 被 ctex 重新计算覆盖。
%  - 将下行的 22pt 改成 24pt / 28pt 即可得到固定行距。


% 行距(统一)
\usepackage{setspace}
\AtBeginDocument{\zihao{-4}} %字号
\setstretch{1.55}   %行距
% 数学公式间距(单独控制)
\AtBeginDocument{%
  \setlength{\abovedisplayskip}{10pt plus 2pt minus 2pt}%
  \setlength{\belowdisplayskip}{10pt plus 2pt minus 2pt}%
}

% \makeatletter
% \renewcommand\normalsize{%
%   \@setfontsize\normalsize{12pt}{20pt}% 小四 + 固定行距 22pt(可改)
%   % 数学公式的上下
%   \abovedisplayskip 10pt plus 2pt minus 2pt
%   \abovedisplayshortskip 8pt plus 2pt minus 2pt
%   \belowdisplayskip \abovedisplayskip
%   \belowdisplayshortskip \abovedisplayshortskip
% }
% \makeatother
% \normalsize

\setlength{\parindent}{2em}   % 首行缩进 2 字符
\setlength{\parskip}{0pt}     % 段前段后:按学校要求可在此调整
